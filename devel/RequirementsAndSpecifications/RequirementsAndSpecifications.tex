\documentclass{scrreprt}
%\usepackage{float}
\usepackage{fullpage}
\usepackage{caption}
\usepackage{subcaption}
\usepackage{listings}
\usepackage{underscore}
\usepackage{tabularx}
\usepackage{longtable}
\usepackage{graphicx}
\usepackage[bookmarks=true]{hyperref}
\usepackage{pdfpages}
\usepackage{cite}

\setcounter{secnumdepth}{3}
\setcounter{tocdepth}{3}

\definecolor{airforceblue}{rgb}{0.66, .0, 0.36}

\hypersetup{
%    bookmarks=false,    % show bookmarks bar?
    pdftitle={Software Requirement Specification},    % title
    pdfauthor={Homer J. Simpson},                     % author
    pdfsubject={TeX and LaTeX},                        % subject of the document
    pdfkeywords={TeX, LaTeX, graphics, images}, % list of keywords
    colorlinks=true,       % false: boxed links; true: colored links
    linkcolor=blue,       % color of internal links
    citecolor=black,       % color of links to bibliography
    filecolor=black,        % color of file links
    urlcolor=purple,        % color of external links
    linktoc=page            % only page is linked
}

\def\myversion{1.0 }

\newcounter{myCounter}[subsubsection] 
\newcounter{mySubCounter}[myCounter] 

\makeatletter
\newcommand{\reqLabel}[1]{%
\myLabel{#1}{Req}}
\makeatother

\makeatletter
\newcommand{\reqLabelB}[1]{%
\myLabelB{#1}{Req}}
\makeatother

\makeatletter
\newcommand{\specLabel}[1]{%
\myLabel{#1}{Spec}}
\makeatother

\makeatletter
\newcommand{\specLabelB}[1]{%
\myLabelB{#1}{Spec}}
\makeatother

\makeatletter
\newcommand{\defLabel}[1]{%
\myLabel{#1}{Def}}
\makeatother

\makeatletter
\newcommand{\nfLabel}[1]{%
\myLabel{#1}{NF}}
\makeatother

\makeatletter
\newcommand{\rightMouse}{%
\includegraphics[width=15pt]{img/rightMouse.png}}
\makeatother

\makeatletter
\newcommand{\rightArrow}{%
\includegraphics[width=10pt]{img/rightArrow.jpeg}\hspace{1mm}}
\makeatother

%%%%%%%%%%%%%%%%%%%%%%%%%%%%%%%%%% Variante 1 %%%%%%%%%%%%%%%%%%%%%%%%%%%%%%%%%%%%%%%
\newcommand{\layerOne}[1]{\chapter{#1}}
\newcommand{\layerOneStar}[1]{\chapter*{#1}}
\newcommand{\layerTwo}[1]{\section{#1}}
\newcommand{\layerThree}[1]{\subsection{#1}}
\newcommand{\layerFour}[1]{\subsubsection{#1}}

\makeatletter
	\newcommand{\myLabel}[2]{%
		\refstepcounter{myCounter}
		\def\@currentlabel{#2-\thesubsubsection.\arabic{myCounter}}% Update label
		\raisebox{\f@size pt}\phantomsection
		\label{req:#1}
		#2-\thesubsubsection.\arabic{myCounter}}
\makeatother

%\makeatletter
%	\newcommand{\myLabelFour}[2]{%
%		\refstepcounter{myCounter}
%		\def\@currentlabel{#2-\thesubsection.\arabic{myCounter}}% Update label
%		\raisebox{\f@size pt}\phantomsection
%		\label{req:#1}
%		#2-\thesubsection.\arabic{myCounter}}
%\makeatother

\newcommand{\kmh}{$kmh^{-1}$}

\makeatletter
	\newcommand{\myLabelB}[2]{%
		\refstepcounter{mySubCounter}
		\def\@currentlabel{#2-\thesubsubsection.\arabic{myCounter}\alph{mySubCounter}}%
		% Update label
		\raisebox{\f@size pt}\phantomsection
		\label{req:#1}
		#2-\thesubsubsection.\arabic{myCounter}\alph{mySubCounter}}
\makeatother

%%%%%%%%%%%%%%%%%%%%%%%%%%%%%%%%%% Variante 2 %%%%%%%%%%%%%%%%%%%%%%%%%%%%%%%%%%%%%%%
%
%\newcommand{\layerOne}[1]{\part{#1}}
%\newcommand{\layerOneStar}[1]{\part*{#1}}
%\newcommand{\layerTwo}[1]{\chapter{#1}}
%\newcommand{\layerThree}[1]{\section{#1}}
%
%\makeatletter
%	\newcommand{\myLabel}[2]{%
%		\refstepcounter{myCounter}
%		\def\@currentlabel{#2-\thesubsection.\arabic{myCounter}}% Update label
%		\raisebox{\f@size pt}\phantomsection
%		\label{req:#1}
%		#2-\thesubsection.\arabic{myCounter}}
%\makeatother
%
%\makeatletter
%	\newcommand{\myLabelB}[2]{%
%		\refstepcounter{mySubCounter}
%		\def\@currentlabel{#2-\thesubsection.\arabic{myCounter}\alph{mySubCounter}}% Update label
%		\raisebox{\f@size pt}\phantomsection
%		\label{req:#1}
%		#2-\thesubsection.\arabic{myCounter}\alph{mySubCounter}}
%\makeatother
%
%%%%%%%%%%%%%%%%%%%%%%%%%%%%%%%%%% Variante Ende %%%%%%%%%%%%%%%%%%%%%%%%%%%%%%%%%%%%%%%

\makeatletter
  \newcommand{\myRef}[1]{
  \ref{req:#1}}
\makeatother

\makeatletter
  \newcommand{\comment}{
  \hspace{1em} \textit{Comment}}
\makeatother

\makeatletter
  \newcommand{\issue}[1]{
  \href{https://github.com/xmf-xmodeler/Mosaic/issues/#1}{Git Issue \##1}}
\makeatother
\title{
\flushright
\rule{15.5cm}{5pt}\vskip1cm
\Huge{SOFTWARE REQUIREMENTS\\ SPECIFICATION}\\
\vspace{2cm}
for\\
\vspace{2cm}
XModeler\\
\vspace{2cm}
\LARGE{Release 1.0\\}
\vspace{2cm}
\LARGE{Version \myversion approved\\}
\vfill
\rule{15.5cm}{5pt}
}
\date{}
\usepackage{hyperref}
\begin{document}
\maketitle
%\includepdf[pages=1]{img/title.pdf}
\newpage
\phantomsection
\addcontentsline{toc}{chapter}{Contents}

\tableofcontents

\clearpage
\phantomsection
\addcontentsline{toc}{chapter}{Revision History}
\layerOneStar{Revision History}

\layerOne{Requirements} 

\begin{tabularx}{\textwidth}[t]{|l|X|} \hline
\reqLabel{general:r1} & Some requirements may be optional. \myRef{general:s1}\\ \hline
\end{tabularx}

\layerTwo{MeMo-Language}

\begin{tabularx}{\textwidth}[t]{|l|X|} \hline
\reqLabel{MeMo:General:r1} & 
The MultiLevel language MeMo requires\ldots\\
\hline
\reqLabel{MeMo:General:r2} & 
The MultiLevel language MeMo requires\ldots\\
\hline
\reqLabel{MeMo:General:r3} & 
The MultiLevel language MeMo requires\ldots\\
\hline
\end{tabularx}

\layerTwo{Kernel}
\layerTwo{X-Modeler}

\layerThree{Diagrams}

\layerFour{Multi-Level Diagram}

\layerThree{Forms}

\layerFour{FormsClient}

\begin{tabularx}{\textwidth}[t]{|l|X|} \hline
\reqLabel{Forms:FormsClient:r1} & 
The FormsClient receives messages to add components.\\
\hline
\reqLabel{Forms:FormsClient:r2} & 
The FormsClient must be able to display these components:
\begin{itemize}
  \item Label
  \item Textfield
  \item Textarea
  \item Checkbox
\end{itemize}
The component is determined by the type of the value.\\
\hline
\reqLabelB{Forms:FormsClient:r2a} & 
A boolean is shown as a Checkbox\\\hline
\reqLabelB{Forms:FormsClient:r2b} & 
An enum is shown as a drop-down-list\\\hline
\reqLabelB{Forms:FormsClient:r2c} & 
A short(?define) Strings or a number is shown as Textfield\\\hline
\reqLabelB{Forms:FormsClient:r2b} & 
A long(?define) String is shown as Textarea\\\hline
\reqLabel{Forms:FormsClient:r3} & 
Labels can be grouped with other components to form a key-value-pair.\\
\hline
\reqLabel{Forms:FormsClient:r4} & 
A form client has a listener for changes which will be transferred instantly to
the underlying model. There are no Save, Cancel or Undo buttons.\\\hline
\reqLabel{Forms:FormsClient:r5} & 
The FormsClient is linked to the underlying model.\\\hline
\reqLabel{Forms:FormsClient:r6} & 
The FormsClient's components are linked to the underlying model's parts
by an id.\\\hline
\comment & Labels should not be empty. \issue{34}\\\hline
\comment & Double click should not freeze the form. \issue{33}\\\hline
\end{tabularx}

\layerOne{Specifications} 

\setcounter{myCounter}{0}
\begin{tabularx}{\textwidth}[t]{|l|X|} \hline
\specLabel{general:s1} & Some specifications may be optional.
\myRef{general:r1}\\
\hline
\end{tabularx}

\layerTwo{XMF-Language}
\layerTwo{Kernel}
\layerTwo{X-Modeler}

\layerThree{Diagrams}

\layerFour{Multi-Level Diagram}

%\layerThree{Signals}
%
%\begin{tabularx}{\textwidth}[t]{|l|X|}
%	  \hline
%    \specLabel{signal:hp0} & Hp0: Stop aspect. One or two red lights. See
    % Fig.~\ref{fig:hp012}, \myRef{signal:1}, \myRef{signal:1a} \\	\hline
%		\specLabel{signal:hp1} & Hp1: Proceed aspect. One green light. See
		% Fig.~\ref{fig:hp012}, \myRef{signal:1}\\ \hline
%		\specLabel{signal:hp2} & Hp2: Slow aspect. One green over one yellow light.
		% Shown for speeds up to $60kmh^{-1}$. If no speed is specified by Zs3 the limit is $40kmh^{-1}$.
%		This default speed limit may be changed to $50kmh^{-1}$ or $60kmh^{-1}$. 
%		See Fig.~\ref{fig:hp012} \\ \hline
%\end{tabularx}
%
%\vspace{.5cm}
%\begin{figure}[ht!]
%	\centering
%	\includegraphics[width=420pt]{img/hp012.pdf}
%	\caption{from left to right: Hp0, Hp1, Hp2}
%	\label{fig:hp012}
%\end{figure} 

%\noindent

\layerOne{Other}

\layerTwo{git}

\includegraphics[width=120pt]{img/git-IconDecorations.png} %% von http://wiki.eclipse.org/EGit/User_Guide

\begin{itemize}
  \item \textbf{dirty (folder)} - At least one file below the folder is dirty; that
  means that it has changes in the working tree that are neither in the index nor in the repository.
  \item \textbf{tracked} - The resource is known to the Git repository and hence
  under version control.
  \item \textbf{untracked} - The resource is not known to the Git repository
  and will not be version controlled until it is explicitly added.
  \item \textbf{ignored} - The resource is ignored by the Git team provider. The
  preference settings under Team $>$ Ignored Resources, "derived" flag and
  settings from .gitignore files are taken into account.
  \item \textbf{dirty} - The resource has changes in the working tree that are
  neither in the index nor in the repository.
  \item \textbf{staged} - The resource has changes which have been added to the
  index. Note that adding changes to the index is currently possible only in the commit dialog via the context menu of a resource.
  \item \textbf{partially-staged} - The resource has changes which are added to the
  index and additional changes in the working tree that neither reached the index nor have been committed to the repository. See partial staging from the Git Staging view for how to do that.
  \item \textbf{added} - The resource has not yet reached any commit in the
  repository but has been freshly added to the Git repository in order to be tracked in future.
  \item \textbf{removed} - The resource is staged for removal from the Git
  repository.
  \item \textbf{conflict} - A merge conflict exists for the file.
  \item \textbf{assume-valid} - The resource has the "assume unchanged" flag. This
  means that Git stops checking the working tree files for possible modifications, so you need to manually unset the bit to tell Git when you change the working tree file. Also see Assume unchanged action.
\end{itemize}

\layerThree{fetch}

Fetch from remote repository

\layerThree{pull}

Incorporates changes from a remote repository into the current branch. 
In its default mode, git pull is shorthand for git fetch followed by git merge FETCH_HEAD.

\layerThree{commit}

Commit changes to local repository

\layerThree{push}

Push Commits to remote repository

\layerThree{conflicts}

\url{http://wiki.eclipse.org/EGit/User_Guide#Resolving_a_merge_conflict}

\layerThree{branches}
branch
merge
rebase
cherry-pick

\layerThree{revert}

If a commit which has not yet been pushed has to be undone:

Select folder \rightMouse Team \rightArrow Show in History \\
Note number of commit
\\ then use command line \\
{\ttfamily git revert <noted number>}

\vspace{5mm}
\noindent If a commit which has already been pushed has to be undone:

Is done the same way.

\vspace{5mm}
\noindent Note: Both reverts don't undo history. The wrong commit and the reversion will
be logged.

\layerOne{Literature}

\begin{itemize}
  \item A Unifying Approach to Connections for Multi-Level Modeling
  \cite{article:AtkinsonGerbigKuehne2015}
  \begin{itemize}
  \item Melanie: Multi-level Modeling and Ontology Engineering
  Environment \cite{article:AtkinsonGerbig2012}
  \item Referenced 2
  \item Referenced 3
\end{itemize}
\end{itemize}

\nocite{article:test}

\layerOne{Weekly Reports}
\setcounter{section}{14}
\layerTwo{2015}
\setcounter{subsection}{36}
\layerThree{7 Sep 2015 - 11 Sep 2015}
\begin{itemize}
\item Made browser tabs closeable
\item Init Requirements and Specifications in LaTeX
\item Added Panic button to GUI
\item Image loading: XML file now derived from IMG file name and path instead of storing
that information in the IMG file.
\item Division of BigIntegers fixed, Multiplication of negative Integers fixed
\end{itemize}
\layerThree{14 Sep 2015 - 18 Sep 2015}
\begin{itemize}
\item Tested and documented some git features.
\item Produced con
icts in git: Managed to resolve them finally. It was challenging. Further
testing required.
\item Tried to install safari. Waiting for Help desk. . .
\item Made the browser work again. Several fixes: Removed layouter, changed locking mechanism,
set to native browser
\end{itemize}
\layerThree{21 Sep 2015 - 25 Sep 2015}
\begin{itemize}
\item Used XMF to do exercises with metaclasses
\end{itemize}
\layerThree{ToDo}
\begin{itemize}
\item improve webpage
\item Specify FormClient
\item git commands in Eclipse
\item (nextWeek) MeMo language requirements
\item pw protected folder: \url{https://wincent.com/wiki/git_repository_access_control}
\item Snippets
\item look at/play with Kernel (Object/Class/Attribute)
\item Use Notepad++ with syntax highlighting for user defined languages: \url{http://weblogs.asp.net/jongalloway/creating-a-user-defined-language-in-notepad}
\item HTMLViewer.xmf: Are ".o", ".xip", ".xto", ".xtd", ".xtml" still in use?
Welcome.xmf: requestURL. . .
requirements
\end{itemize}


\clearpage
\phantomsection
\addcontentsline{toc}{chapter}{Bibliography}
\bibliography{literature}
\bibliographystyle{literatureStyle}
\end{document}
